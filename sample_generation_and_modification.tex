\documentclass[Thesis.tex]{subfiles}
\begin{document}

\section{Sample generation and modification}

One important part of the algorithm is sampling. Not only the sampling of new poses is crucial, it is also important that the noise in the motion model is sampled correctly. 

Therefore the algorithm uses three different sampling strategies, depending on the use case for the sampled values. The first strategy is a uniform sampling to generate the initial poses. The second strategy (described on \pageRef{sec:motion_model_section}) is used for sampling noise according to the robot's sensor precision to simulate an accurate motion model. The last strategy is used to determine which poses need to be removed and where to regenerate new samples: this process filters the poses by replacing unlikely pose samples by pose samples with a bias towards the most likely ones.

\subsection{Generating initial pose samples}\label{sec:init_samples}
The generation of new pose samples is done by drawing random numbers from a uniform distribution. For the position there is no more work needed than mapping the sampled values into the correct range. For a random value $v$ from the range $S$ the remapped value $v_n$ from the range $T$ gets calculated by the formula $v_n = T_{min} + v \frac{|S|}{|T|}$, where $|S|$ denotes the length of range $S$ (or $T$, respectively) and $T_{min}$ denotes the smallest value of $T$. For the three needed values $x, y,$ and $z$ this results in the following calculations.
%
\begin{align}
x &= x_{min} + \textbf{rand()} \left( x_{max} - x_{min} \right) \\
y &= y_{min} + \textbf{rand()} \left( y_{max} - y_{min} \right) \\
z &= z_{min} + \textbf{rand()} \left( z_{max} - z_{min} \right) \\
v &= (x, y, z)
\end{align}

In this equation ${\bf rand()}$ draws \gls{iid} random numbers from the source range $R_s = \left[0, \dots, 1\right]$. $x_{min}$ and $x_{max}$ (for $y$ and $z$ respectively) describe the target range $R_t = \left[x_{min}, \dots, x_{max}\right]$ in which the samples shall be generated for the respective dimension. In this thesis they correspond to the boundaries of the map in each dimension.

By subtracting the minimum value of the source range $R_s$ from the sampled value, the range the random value is drawn from is shifted to zero---in this case the subtracted value would be $0$, so it's left out. The fraction $\frac{ x_{max} - x_{min} }{ 1 - 0 }$ describes the ratio between the lengths of the two ranges $\frac{|R_t|}{|R_s|}$. Again this can be simplified to just $x_{max}-x_{min}$. A multiplication of the shifted value and the ratio maps the value to the correct value in a zero shifted target interval with the length of the target range $|R_t|$. By adding the target range's minimal value, $x_{min}$, the sampled value gets shifted to the correct range.

This procedure is done for each of the three positional coordinates, resulting in the positional vector $v = \left(x, y, z\right)$.

Sampling the orientation is done with the \emph{subgroup algorithm} by Ken Shoemake\cite[p.~129-130]{gfxgems:1995}. It generates a unit quaternion from three uniformly \gls{iid} random values $X_0, X_1, X_2$ of the following form:
%
\begin{align}
q = \left( \sin{2\pi X_1}\sqrt{1-X_0},\: \cos{2\pi X_1}\sqrt{1-X_0},\: 
           \sin{2\pi X_2}\sqrt{  X_0},\: \cos{2\pi X_2}\sqrt{  X_0} \right)
\end{align}

%\begin{algorithm}
%\KwIn{Random variables $X_0, X_1, X_2$ between 0 and 1} 
%\KwOut{Unit quaternion $q$}
%$\theta_1 = 2\pi X_1$\;
%$\theta_2 = 2\pi X_2$\;
%$s_1 = \sin{2\pi X_1}$\;
%$s_2 = \sin{2\pi X_2}$\;
%$c_1 = \cos{2\pi X_1}$\;
%$c_2 = \cos{2\pi X_2}$\;
%$r_1 = \sqrt{1-X_0}$\;
%$r_2 = \sqrt{X_0}$\;
%\Return{q($s_1r_1, c_1r_1, s_2r_2, c_2r_2$)}
%\end{algorithm}

The generated vector $v$ and the orientation quaternion $q$ form the sampled robot pose. Initially all poses get the same probability: $\nicefrac{1}{n}$, where $n$ is the number of samples.

\subsection{Regenerating samples}
\todo[inline]{this has to be implemented yet}
Sample regeneration is performed as the last step in the \gls{AMCL6D} algorithm. The method used for resampling individual particles depends on $\theta_{close}, \theta_{random},$ and $\theta_{prob}$.

After each iteration 

\end{document}