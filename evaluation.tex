\documentclass[Thesis.tex]{subfiles}
\begin{document}
\chapter{Simulation and results}

The \gls{ROS} node built for \gls{AMCL6D} also includes a simple tool to simulate a robot pose. This tool was used to simulate and test the algorithm.
To test the performance of the algorithm it was tested in the following categories:
\begin{itemize}
	\item The resolution of the ray tracer,
	\item The performance in different maps,
	\item and the performance in the same maps with different level of details.
\end{itemize}
To assure accuracy, for each map there was a random but arbitrary sequence of motions defined which was repeated several times. The simulated camera for the ray tracing always had aperture angles of $90{\ }^\circ$ (horizontal) and $60{\ }^\circ$ (vertical). 

\section{Resolution of ray tracer}
For the tests which resolutions yield good results


\section{Performance in different maps}
The performance is measured by taking the distance between the simulated real pose and the best hypothesis. For each of them the best values are stored.

\subsection{Elevator doors}
\begin{tabular}{lr}
  \multicolumn{2}{c}{\bf Map/Test properties} \\
	Vertex count & 3,774 \\
  Face count & 3,924 \\
  Sequence size & 100 \\
  Resolution & 65x45
\end{tabular}

{\renewcommand{\arraystretch}{0.5}
\noindent\begin{tabular}{@{\hspace{-2pt}}r|@{}r@{}r@{}r|@{}r@{}r@{}r|@{}r@{}r|@{}r@{}r}
& \multicolumn{3}{c|}{\bf Best Position} 
& \multicolumn{3}{c|}{\bf Best Orientation} 
& \multicolumn{2}{c|}{\bf Termination}
& \multicolumn{2}{c}{\bf Time (s)} \\
\# & Pos.dst. & Or.dst. & Iter. &  Pos.dst. & Or.dst. & Iter. &  Pos.dst. & Or.dst. & Total & Avg.  \\ \toprule
1   & 1.207   & 1.003   &    1  &  10.810   & 1.000   &   84  &  10.778   & 1.079   & 2139  & 21.39 \\
    &         &         &       &           &         &       &           &         &       &       \\
    &         &         &       &           &         &       &           &         &       &       \\
    &         &         &       &           &         &       &           &         &       &       \\
    &         &         &       &           &         &       &           &         &       &       \\
\end{tabular}}

\subsection{Church}


\subsection{Police station}


\subsection{Office corridor}



\section{Ray tracing speed}
Since the ray tracer was considered being slow during development, the time of the ray tracer was tracked during the other experiments as well.
The results show what was already expected: The ray tracer consumes most of the node's running time and is a huge bottleneck.

\noindent\begin{tabular}{r|lc|rr|l}
\bf Run & \bf Map        & \bf Resolution & \bf Total time (s) & \bf RT time (s) & \bf \nicefrac{RT}{Total} (\%) \\ \toprule
    1   & Elevator doors &  65x45         & 2139               & 2037            & 95.23 \\
    2   & Elevator doors &  65x45         & 2191               & 2085            & 95.16 \\
        & Elevator doors &  65x45         &                    &                 &       \\
        & Elevator doors &  65x45         &                    &                 &       \\
        & Elevator doors &  65x45         &                    &                 &       \\
\bottomrule
\end{tabular}





\end{document}
