\documentclass[../Thesis.tex]{subfiles}
\begin{document}

\chapter{Idea and Motivation}

There are several algorithms which solve localization problems, be they local or global, in two dimensional maps. This is useful for many real-life applications, since lots of robots operate in more or less flat environments.

Although it is sufficient for many environments to represent them as a 2D map, environments with altitude differences (e.g. factories with ramps or normal houses with stairs) are hard to represent this way. 


Common representations for such environments are meshes, i.e. vertices connected to faces. 

\todo[inline]{Add something}

In this thesis I will derive an algorithm to localize a robot in an arbitrary 3D environment which is represented by a continuous 3D map. 

With the extension to a third dimension the environment should offer fewer ambiguous locations than two dimensional ones, since there are more features to detect and to be compared with the map. This should lead to a faster converge towards the real position of the robot.





\end{document}